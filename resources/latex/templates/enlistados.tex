% Remission note template (Patient's version)
% ===========================================
% Version 0.2
% For splitting in half: http://tex.stackexchange.com/questions/4243/spliting-a-page-in-half-horizontally

\documentclass[12pt]{article}
\usepackage[utf8]{inputenc}
\usepackage[spanish]{babel}
\usepackage{fancyhdr}
\usepackage{graphicx}
\usepackage{color}
\usepackage{longtable}
\usepackage[table]{xcolor}
\pagestyle{fancy}

% Header config
\headheight = 55pt
\renewcommand{\headrulewidth}{0pt}
\lhead{\textbf{Beatriz Mayoral}\\ \scriptsize{R.F.C MAPB-590708-MF0 \\ Alfonso Ortiz Tirado No. 16-A \\ Col. Centro, Hermosillo, Sonora}}
\rhead{\includegraphics[height=50pt]{../../../../img/logo-optica-large.png}}
\cfoot{}

\footskip = 0pt

\begin{document}
	\begin{center}\textbf{#header#}\end{center}

	\renewcommand{\arraystretch}{1.3}
	\begin{center}
		{\footnotesize
			\rowcolors{2}{white}{lightgray}
			\begin{longtable}{l l l}
				\textbf{Nombre} & \textbf{Apellido} & \textbf{Diagnóstico}\\ \hline \hline
				#diagnostics_list#
				\hline
			\end{longtable}
		}
	\end{center}

	\textbf{Resumen de diagnosticos}
	\begin{center}
		\rowcolors{2}{white}{lightgray}
		\begin{tabular}{l p{1.5cm} p{1.5cm}}
			\textbf{Total de personas revisadas\hspace{1.5cm}} & #num_patients# & \\
			\hspace{1cm} \textbf{Ametropías} & & \\
			\hspace{2cm} Astigmatismo & #astigmatismo_n# & #astigmatismo_p#\% \\
			\hspace{2cm} Miopía & #miopia_n# & #miopia_p#\% \\
			\hspace{2cm} Hipermetropía & #hipermetropia_n# & #hipermetropia_p#\% \\
			\hspace{1cm} \textbf{Presbícia} & #presbicia_n# & #presbicia_p#\% \\
			\hspace{1cm} \textbf{Anisometropía} & #anisometropia_n# & #anisometropia_p#\% \\
		\end{tabular}
	\end{center}

\end{document}
